\documentclass[]{article}
\usepackage[T1]{fontenc}
\usepackage{lmodern}
\usepackage{amssymb,amsmath}
\usepackage{ifxetex,ifluatex}
\usepackage{fixltx2e} % provides \textsubscript
% use microtype if available
\IfFileExists{microtype.sty}{\usepackage{microtype}}{}
% use upquote if available, for straight quotes in verbatim environments
\IfFileExists{upquote.sty}{\usepackage{upquote}}{}
\ifnum 0\ifxetex 1\fi\ifluatex 1\fi=0 % if pdftex
  \usepackage[utf8]{inputenc}
\else % if luatex or xelatex
  \usepackage{fontspec}
  \ifxetex
    \usepackage{xltxtra,xunicode}
  \fi
  \defaultfontfeatures{Mapping=tex-text,Scale=MatchLowercase}
  \newcommand{\euro}{€}
\fi
\ifxetex
  \usepackage[setpagesize=false, % page size defined by xetex
              unicode=false, % unicode breaks when used with xetex
              xetex]{hyperref}
\else
  \usepackage[unicode=true]{hyperref}
\fi
\hypersetup{breaklinks=true,
            bookmarks=true,
            pdfauthor={},
            pdftitle={Analyzing Cooperation with Game Theory and Simulation},
            colorlinks=true,
            urlcolor=blue,
            linkcolor=magenta,
            pdfborder={0 0 0}}
\urlstyle{same}  % don't use monospace font for urls
\setlength{\parindent}{0pt}
\setlength{\parskip}{6pt plus 2pt minus 1pt}
\setlength{\emergencystretch}{3em}  % prevent overfull lines
\setcounter{secnumdepth}{0}

\title{Analyzing Cooperation with Game Theory and Simulation}
\author{}
\date{}

\begin{document}
\maketitle

\section{Analyzing Cooperation with Game Theory and Simulation}

\section{Tutorial for the R package StratTourn}

\textbf{Date: 2013-10-21}

\textbf{Author: Sebastian Kranz
(\href{mailto:sebastian.kranz@uni-ulm.de}{sebastian.kranz@uni-ulm.de})}

\textbf{with help by Martin Kies
(\href{mailto:martin.kies@uni-ulm.de}{martin.kies@uni-ulm.de})}

\subsection{1. Installing neccessary software}

\subsubsection{1.1 Installing R and RStudio}

First you need to install R, which is a very popular and powerful open
source statistical programming language. You can download R for Windows,
Max or Linux here:

\href{http://cran.r-project.org/}{http://cran.r-project.org/}

Note: If you have already installed R, you may want to update to the
newest version by installing it again.

I recommend to additionally install RStudio, which is a great open
source IDE for R:

\href{http://rstudio.org/}{http://rstudio.org/}

\subsubsection{1.2 Installing necessary R packages}

You need to install several R packages from the internet. To do so,
simply run in the R console the following code (you can use copy \&
paste):

\begin{verbatim}
install.packages("devtools", "data.table", "ggplot2", "reshape2")

library(devtools)
install_github(repo = "restorepoint", username = "skranz")
install_github(repo = "sktools", username = "skranz")
install_github(repo = "StratTourn", username = "skranz")
\end{verbatim}

\subsection{2. Brief Background: The Prisoners' Dilemma Game}

In the `Evolution of Cooperation' Robert Axelrod describes his famous
computer tournament that he used to investigate effective cooperation
strategies in repeated social dilemma situations. His tournament was
based on a repeated \emph{prisoner's dilemma game}, which I will also
use as an example in this tutorial.

\subsubsection{2.1 The one-shot Prisoner's Dilemma}

A one-shot prisoner's dilemma (short: PD) is played only once. We can
describe a PD by the following \emph{payoff matrix}:

Pl 1. / Pl. 2

C

D

C

1, 1

-1, 2

D

2,-1

0, 0

Each player can either cooperate \emph{C} or defect \emph{D}, which
means not to cooperate. The row shows player 1's action and the column
player 2's. The cells show the \emph{payoffs} of player 1 and 2 for each
\emph{action profile}. For example, when player 1 cooperates C and
player 2 defects D, i.e. the action profile (C,D) is played, then player
1's payoff is -1 and player 2's payoff is 2.

The highest average payoff (1,1) could be achieved if both players
cooperate (C,C). Yet, assume that both players decide independently
which action they choose, i.e. player 1's action does not influence
player 2's action and vice versa. Then no matter which action the other
player chooses, a player always maximizes its own payoff by choosing D.
In game theoretic terms, choosing D is \emph{strictly dominant} in a PD
game that is played only once. (Consequently, the only \emph{Nash
equilibrium} of the one-shot PD is (D,D)).

\subsubsection{2.2 The repeated Prisoner's Dilemma}

In this seminar, we will mostly consider games that are repeated for an
uncertain number of periods. Let \textbackslash{}( \textbackslash{}delta
\textbackslash{}in {[}0,1) \textbackslash{}) be a continuation
probability (we will also often refer to \textbackslash{}(
\textbackslash{}delta \textbackslash{}) as \emph{discount factor}).
After each period a random number is drawn and the repeated game ends
with probability \textbackslash{}( 1-\textbackslash{}delta
\textbackslash{}), i.e. it continues with probability \textbackslash{}(
\textbackslash{}delta \textbackslash{}). In the basic version (formally
called a PD game with \emph{perfect monitoring}) all players exactly
observe the actions that the other players have played in the previous
periods. It is then no longer strictly dominant to always play D. For
example, player 2's action in period 2 may depend on what player 1 did
in period 1: she may only play C if player 1 has played C in period 1
and otherwise play D. A player's optimal strategy now depends on how the
other player reacts to her action and cooperation may become rational
even for pure egoists.

It turns out that if the continuation probability \textbackslash{}(
\textbackslash{}delta \textbackslash{}) is large enough, such repeated
games often have a very large number of \emph{Nash equilibria}. This
means we can have many possible profiles of strategies that are stable
in the sense that it is optimal for each player to follow her strategy
given the strategies of the other players. In some Nash equilibria
players will cooperate a lot, in others very little or not at all.

\subsubsection{2.3 What will we do in the seminar?}

We search for stable, socially efficient cooperative strategy profiles
in several strategic interactions and study which factors make
cooperation harder and how one best adapts cooperation strategies to
such factors. I will typically interpret such strategy profiles as
stable \emph{social norms} that allow a high degree of cooperation in a
society.

We do this search by programming strategies in R and let them play
against each other in a tournament (more on that below). During the
seminar, we will discuss a bit informally in how far the winning
strategies of our tournament are related to game theoretic equilibrium
concepts, like the \emph{Nash equilibrium}. For those who already know
more about Nash equilibria, let us just state that the socially most
efficient Nash equilibria probably would have a very good chance to be
winners in our tournaments, but often it is very hard to find these
equilibria.

\subsection{3. Getting Started: Developing strategies for the repeated
Prisoner's Dilemma in R}

Preparations:

\begin{enumerate}
\itemsep1pt\parskip0pt\parsep0pt
\item
  Make sure you installed all software and packages as described in
  Section 1
\item
  Download the file \emph{pdgame.r}, save it in some working directory,
  and open it with RStudio.
\item
  Press the ``Source'' button above on the right of the editor window to
  load all functions of this file.
\item
  Consider the code inside the function examples.pd, select the
  following lines and press the ``Run'' button (or press Ctrl-Enter):
\end{enumerate}

\begin{verbatim}
# Load package
library(StratTourn)

# Generate a PD game object
game = make.pd.game()
# Pick a pair of strategies (strategies are defined below)
strat = nlist(tit.for.tat, random.action)

# Let the strategies play against each other one repeated PD
run.rep.game(delta = 0.9, game = game, strat = strat)
\end{verbatim}

\begin{verbatim}
## $hist
##   obs_a1 obs_a2 a1 a2 pi1 pi2
## 1   <NA>   <NA>  C  C   1   1
## 2      C      C  C  C   1   1
## 3      C      C  C  D  -1   2
## 4      C      D  D  D   0   0
## 5      D      D  D  D   0   0
## 6      D      D  D  C   2  -1
## 7      D      C  C  C   1   1
## 8      C      C  C  D  -1   2
## 9      C      D  D  D   0   0
## 
## $u
## [1] 0.3333 0.6667
\end{verbatim}

This code simulates a repated PD with continuation probability
\textbackslash{}( \textbackslash{}delta=0.9 \textbackslash{}) in which
player 1 follows a strategy called ``tit.for.tat'' and player 2 follows
a strategy called ``random.action''. The resulting table shows for each
period the following information:

\begin{itemize}
\itemsep1pt\parskip0pt\parsep0pt
\item
  Columns obs\_a1 and obs\_a2: the observations at the beginning of a
  period. Here just the previous actions
\item
  Columns a1 and a2: players' actions
\item
  Columns pi1 and pi2: the resulting payoffs in that period.
\end{itemize}

Below the table, the entry \$u shows players' average payoffs across all
periods.

\subsubsection{3.1 Tit-for-Tat}

Tit-for-Tat was the winning strategy in Axelrod's original tournament.
It has a simple structure:

\begin{itemize}
\itemsep1pt\parskip0pt\parsep0pt
\item
  Start nice by cooperating in period 1
\item
  In later periods play that action that the other player has played in
  the previous period.
\end{itemize}

\subsubsection{3.2 Tit-for-Tat as a R function}

Further below in the file pdgame.r you find a definition of tit.for.tat
as a R function:

\begin{verbatim}
tit.for.tat = function(obs, i, t, game) {
    debug.store("tit.for.tat", i, t)  # Store each call for each player
    debug.restore("tit.for.tat", i = 1, t = 2)  # Restore call for player i in period t

    # Cooperate in the first period
    if (t == 1) 
        return(list(a = "C"))

    # In later periods, return the other player's previous action
    j = 3 - i
    list(a = obs$a[j])
}
\end{verbatim}

The first line

\begin{verbatim}
tit.for.tat = function(obs,i,t,game) {
\end{verbatim}

says that tit.for.tat is a function with the following arguments:

\begin{itemize}
\itemsep1pt\parskip0pt\parsep0pt
\item
  i: this is simply the number of the player, either 1 or 2
\item
  t: this is the number of the current period
\item
  obs: this is a
  \href{http://cran.r-project.org/doc/manuals/R-intro.html\#Lists}{list}
  that contains a player's observations about behavior from the previous
  period. The exact structure of observations will depend on the
  specification of the game that is played. In our basic PD game obs
  contains an element obs\$a that is a vector with the two actions that
  the players have chosen in the previous period.
\item
  game: this is an object that describes the structure of the game that
  is currently played. In game\$param different parameters of the game,
  like the stage game payoffs, are stored. You only need to use this
  variable, if you want to write general strategies that can be used for
  more than one game and that are fine tuned for different games. In the
  moment you just can ignore this variable.
\end{itemize}

Every strategy must have these 4 arguments. (There may be additional
aguments if a strategy uses states to pass additional information
between periods. This is explained further below). The function
run.rep.game now calls this function tit.for.tat in every period and
provides the corresponding values of i,t,obs and game. E.g. if it is
called in the third period for player 1 and in the previous round (D,C)
was played, we have i==1, t==3, obs\$a{[}1{]}==``D'' and
obs\$a{[}2{]}==``C''. Based on the values of obs, t, i the function must
return the action that the player chooses.

The lines

\begin{verbatim}
debug.store("tit.for.tat", i, t)  # Store each call for each player
debug.restore("tit.for.tat", i = 1, t = 2)  # Restore call for player i in period t
\end{verbatim}

are useful for debugging a strategy and can be ignored for the moment
(you can also remove them without changing the behavior of the
strategy).

The lines

\begin{verbatim}
if (t == 1) return(list(a = "C"))
\end{verbatim}

state that in the first period, i.e. t==1, the player cooperates. That
the player cooperates means that the function returns a list

\begin{verbatim}
list(a = "C")
\end{verbatim}

where the element a is ``C''. In more complex games, a function may
return more than a single action. The exact structure of the list of
actions that a function has to return will be explained in the
corresponding exercises. Note that we ommitted the ``\{'' brackets of
the `if' condition. We can do that iff there follows exactly one
statement after the condition.

The lines

\begin{verbatim}
j = 3 - i
list(a = obs$a[j])
\end{verbatim}

describe the behavior in periods t\textgreater{}1. The variable j will
be the index of the other player (if i=1 then j=2 and if i=2 then j=1).
The last line says that the player choses that action that the other
player has played in the previous period, i.e. obs\$a{[}j{]}. (Note that
in the last line of a function you can ommit writing ``return''.)

\subsubsection{3.3 Strategies that use states. Example:
strange.defector}

Many strategies rely not only on the most recent observations which are
saved in obs. Consider for example this self-developed (not very clever)
strategy:

\begin{itemize}
\itemsep1pt\parskip0pt\parsep0pt
\item
  ``Strange Defector'': In the first round cooperates with 70\%
  probability, otherwise defects. As long as the player cooperates, he
  continues in this random fashion. Once the player defects, he plays 4
  additional times ``defect'' in a row. Afterwards, he plays again as in
  the first period (randomizing, and after defection, 4 defects in a
  row).
\end{itemize}

Here is an R implementation of this strategy:

\begin{verbatim}
strange.defector <- function(obs, i, t, game, still.defect = 0) {
    debug.store("strange.defector", i, t)  # Store each call for each player
    debug.restore("strange.defector", i = 1, t = 2)  # Restore call for player i in period t

    # Randomize between C and D
    if (still.defect == 0) {
        do.cooperate = (runif(1) < 0.7)
        # With 60% probability choose C
        if (do.cooperate) {
            return(list(a = "C", still.defect = 0))
        } else {
            return(list(a = "D", still.defect = 4))
        }
    }

    # still.defect is bigger 0: play D and reduce still.defect by 1
    still.defect = still.defect - 1
    return(list(a = "D", still.defect = still.defect))
}
\end{verbatim}

Compared to the tit.for.tat function, the strange.defector function has
an additional argument, namely \emph{still.defect} which in the first
round t=1 has the value 0. Also the returned lists contain an additional
field named \emph{still.defect}. The variable \emph{still.defect} is a
manually generated \emph{state variable}.

\paragraph{How state variables transfer information between periods:}

\begin{verbatim}
The value of a state variable that is passed to your function in period t is the value of the state that your function has returned in period t-1. (The value of a state in period 1 is the value you specify in the function definition).
\end{verbatim}

\paragraph{Name and number of state variables}

\begin{verbatim}
You can freely pick the name of a state variable (except for the reserved names ops,i,t, game and a) and you can have more than one state variable.
\end{verbatim}

\paragraph{Which values can state variables take?}

\begin{verbatim}
States can take all sort of values: numbers (e.g 4.5), logical values (TRUE or FALSE), strings (e.g. "angry"), or even vectors. You just should not store a list in a state variable.
\end{verbatim}

\paragraph{Back to the example:}

In our example, the state variable \emph{still.defect} captures the
information how many rounds the streak of defection should still last.

Let us have a more detailed look at the code of the example. The line

\begin{verbatim}
  strange.defector <- function(obs, i, t, game, still.defect=0){
\end{verbatim}

initializes the function with a state still.defect that has in the first
period a value of 0. The lines

\begin{verbatim}
if (still.defect == 0) {
    do.cooperate = runif(1) < 0.7
    # With 60% probability choose C
    if (do.cooperate) {
        return(list(a = "C", still.defect = 0))
    } else {
        return(list(a = "D", still.defect = 4))
    }
}
\end{verbatim}

first check whether we are in the initial state (still.defect=0), in
which we randomize between C and D. If this is the case, we draw with
the command

\begin{verbatim}
do.cooperate = (runif(1) < 0.7)
\end{verbatim}

a logical random variable that is TRUE with 70\% probability and
otherwise FALSE. (To see this, note that runif(1) draws one uniformely
distributed random variable between 0 and 1). The lines

\begin{verbatim}
    if (do.cooperate){
      return(list(a="C", still.defect=0))
\end{verbatim}

state that if the random variable says that we should cooperate, we
return the action ``C'' and keep the state still.defect=0. The lines

\begin{verbatim}
    } else {
      return(list(a="D", still.defect=4))
    }
\end{verbatim}

state that otherwise, we return the action ``D'' and set the state
still.defect = 4. This means that we will defect for the next 4 periods.
In the next period the value of still.defect will be 4 and the code at
the bottom of the function will be called:

\begin{verbatim}
still.defect = still.defect - 1
return(list(a = "D", still.defect = still.defect))
\end{verbatim}

The first line reduces still.defect by 1. (Hence, after 4 periods, we
will again be in the state still.defect =0 and choose a random action).
The second line returns our action a=``D'' and the new value of our
state still.defect.

If you run a single repeated game the result table also shows in each
row, the values of the strategies' states at the \emph{end} of the
period:

\begin{verbatim}
run.rep.game(delta = 0.9, game = game, strat = nlist(strange.defector, tit.for.tat), 
    T.min = 20)
\end{verbatim}

\begin{verbatim}
## $hist
##    obs_a1 obs_a2 a1 a2 pi1 pi2 still.defect_1
## 1    <NA>   <NA>  D  C   2  -1              4
## 2       D      C  D  D   0   0              3
## 3       D      D  D  D   0   0              2
## 4       D      D  D  D   0   0              1
## 5       D      D  D  D   0   0              0
## 6       D      D  D  D   0   0              4
## 7       D      D  D  D   0   0              3
## 8       D      D  D  D   0   0              2
## 9       D      D  D  D   0   0              1
## 10      D      D  D  D   0   0              0
## 11      D      D  D  D   0   0              4
## 12      D      D  D  D   0   0              3
## 13      D      D  D  D   0   0              2
## 14      D      D  D  D   0   0              1
## 15      D      D  D  D   0   0              0
## 16      D      D  D  D   0   0              4
## 17      D      D  D  D   0   0              3
## 18      D      D  D  D   0   0              2
## 19      D      D  D  D   0   0              1
## 20      D      D  D  D   0   0              0
## 21      D      D  C  D  -1   2              0
## 22      C      D  D  C   2  -1              4
## 23      D      C  D  D   0   0              3
## 24      D      D  D  D   0   0              2
## 25      D      D  D  D   0   0              1
## 26      D      D  D  D   0   0              0
## 27      D      D  D  D   0   0              4
## 28      D      D  D  D   0   0              3
## 29      D      D  D  D   0   0              2
## 30      D      D  D  D   0   0              1
## 
## $u
## [1]  0.21066 -0.08534
\end{verbatim}

\subsubsection{3.4 Exercise:}

Implement the following strategy in R.

\begin{itemize}
\item
  tit3tat: The player starts with C and plays like tit-for-tat in period
  t=2 and t=3. In period t\textgreater{}3 the player plays with 60\%
  probability like tit-for-tat and with 30\% probability he plays the
  action that the other player played in the pre-previous period, i.e.
  in t-2 and with 10\% probability he plays the action the other player
  played in t-3.

  Hints:
\item
  To do something with 60\% probability, you can draw a random variable
  x with the command x=runif(1) that is uniformely distributed between 0
  and 1 and then check whether x\textless{}=0.6. To do something else
  with 30 probability, you can check 0.6 \textless{} x \& x \textless{}=
  0.9 and so on\ldots{}
\item
  To save a longer history you can either use a function that has more
  than one state or store a vector in a state variable.
\end{itemize}

\subsection{4. Running a tournament between strategies}

The following lines run a tournament between 4 specified strategies

\begin{verbatim}
# Init and run a tournament of several strategies against each other
game = make.pd.game(err.D.prob = 0.15)
strat = nlist(strange.defector, tit.for.tat, always.defect, always.coop)
tourn = init.tournament(game = game, strat = strat, delta = 0.95, score.fun = "efficiency-2*instability- 20*instability^2")
tourn = run.tournament(tourn = tourn, R = 15)
tourn
\end{verbatim}

\begin{verbatim}
## 
## Tournament for Noisy PD (15 rep.)
## 
##                  strange.defector tit.for.tat always.defect always.coop
## strange.defector            0.351      0.3150       -0.3200        1.63
## tit.for.tat                 0.339      0.2870       -0.0499        1.13
## always.defect               0.751      0.0997        0.0000        2.00
## always.coop                -0.264      0.7230       -1.0000        1.00
## 
## Ranking with score = efficiency-2*instability- 20*instability^2
## 
##                  rank   score efficiency instability u.average   best.answer
## always.defect       1   0.000      0.000       0.000     0.713 always.defect
## strange.defector    2  -3.640      0.351       0.399     0.493 always.defect
## tit.for.tat         3  -4.387      0.287       0.436     0.427   always.coop
## always.coop         4 -21.000      1.000       1.000     0.115 always.defect
\end{verbatim}

The tournament consists of R rounds (here R=15). In every round, every
strategy plays against each other strategy in a repeated game. Every
strategy also plays against itself. For every pairing of two strategies,
we then compute the average payoff of the strategies over all R rounds

The first matrix has for every strategy a row and column and shows the
average payoff of the \emph{row strategy} when the row strategy plays
against the column strategy. (In a symmetric game like the PD game, we
will just display the average payoffs of both player 1 and player 2).
For example, when strategy always.defect played against always.coop,
then always.defect got on average a payoff of 2 and always.coop got on
average a payoff of -1. Against itself always.coop got an average payoff
of 1.

The table below shows which strategies have won, as well as their score,
efficiency, instability and best answer.

\subsubsection{4.1 Social norms, efficiency, instability and score}

How do we determine the winner of the tournament? Axelrod ranked the
strategies according to their average payoff across all pairings. This
average payoff is shown in the column \emph{u.average}. We will use
different scores to rank strategies, however.

Our underlying question is the following. Assume interactions in a group
are such that people, who essentially want to maximize their own
payoffs, are randomly matched with each other and play the specified
repeated game. What would be a \textbf{``good social norm''} that one
could teach to all people and which describes how they should behave in
their interactions? More compactly:

** We search for strategies that would be efficient and stable social
norms if everybody would follow these strategies**

What do we mean with efficient and stable?

\paragraph{\textbf{Efficiency:}}

We define efficiency of a strategy is the \textbf{average payoff} the
strategy achieves when it plays \textbf{against itself}. Hence, the
efficiency is the average payoff of every person in the group if
\emph{everybody} in the group would follow this strategy.

\paragraph{\textbf{Best answer:}}

The best answer against a strategy s shall be that strategy (from all
strategies that participate in the tournament) that has the highest
payoff against s. The column \emph{best.answer} shows the best answer
for each strategy. Note that several strategies can be best answers if
they achieve the same payoff (we then only show one). Also note that a
strategy can be a best answer against itself.

\paragraph{\textbf{Instability:}}

We define the instability of a strategy s as the following difference:

\begin{verbatim}
instability = best answer payoff against s - payoff of s against itself
\end{verbatim}

Note that a strategy that is a best answer against itself has an
instability of 0 and is therefore very stable. The following idea lies
behind this definition of instability. We assume that persons want to
maximize their own payoff. If s has a high instability value, it has the
following problem: if everybody follows s, people have very high
incentives not to follow s (deviate from s) and play the best answer
strategy instead. The strategy s would not be a robust, sustainable
social norm.

\paragraph{Relationship to Best Replies and Nash equilibrium}

The idea that strategies should be stable is closely related to the game
theoretic concept of a \textbf{Nash equilibrium}. Similar to our concept
of a best answer, game theory knows the related concept of a \emph{best
reply}:

\begin{itemize}
\item
  In a two player game, player i's \textbf{best reply} to a strategy s
  of player j is that strategy that maximizes i's expected payoff across
  \textbf{all possible strategies}
\item
  If every player follows a strategy s, we have a \textbf{Nash
  equilibrium} if and only if \textbf{s is a best reply against itself}
\end{itemize}

Hence a strategy that forms a \textbf{Nash equilibrium} would have
instability of 0 (abstracting from small sample problems), yet
instability of 0 does not imply that the strategy will form a Nash
equilibrium. That is because we have the following main difference
between best-replies and best answers:

\begin{itemize}
\itemsep1pt\parskip0pt\parsep0pt
\item
  \textbf{Main difference best answer vs best reply}: Best answers only
  consider those strategies that participate in the tournament, while a
  best reply considers \emph{all} possible strategies (there are
  typically infinitely many strategies).
\end{itemize}

There are also additional differences. If you are not firm in game
theory, you just can ignore them. If you are firm you just should notice
these points but hopefully not bother too much.

\begin{itemize}
\item
  To compute best answers, we just take average payoffs from small
  sample of played repeated games, best replies are based on true
  expected payoffs.
\item
  Compared to game theoretic terminology, our use of the term strategy
  is quite imprecise with respect to which players plays the strategy
  (Our R functions are defined for both players and could in game
  theoretic terms also be seen as \emph{strategy profiles}). When
  computing best answers we don't distinguish between player 1's best
  answer against player 2 and vice versa, but we will just take the mean
  of player 1's and player 2's payoffs.
\end{itemize}

\paragraph{Score}

The score of a strategy is a formula that combines its effieciency and
instability. The basic rule is

\begin{verbatim}
The score increases in the efficiency and decreases in the instability
\end{verbatim}

In the example we use the following formula: \textbackslash{}{[} score =
efficiency - 2 * instability - 20*instability\^{}2 \textbackslash{}{]}

The quadratic term makes sure that large instability gets more strongly
penalized than small instability. It is not clear how much we want to
penalize instability. Game theory always puts a lot of emphasis on
stability, since a Nash equilibrium needs an instability of 0. On the
other hand, checking instability is not so easy and it can be very hard
in the tasks of our seminar to find somewhat efficient strategies that
really have an instability of 0. A score that allows some degree of
instability may be a sensible criterion if we think about norms in a
society in which at least some people have some intrinsic motivation to
follow a prevailing norm or simply are too lazy to think much about a
profitable deviation from the prevailing norm.

In our example, always.defect is the winner of the tournament. It is
able to sustain cooperation in every period and has instability 0 (one
can indeed show that it is a Nash equilibrium of that repeated PD).
While always.coop also has a high efficiency, it looses because it is
very instable. That is because a player who plays always.defect against
always.coop makes a much higher payoff than a player who plays
always.coop against always.coop.

\paragraph{Second stage of tournament: Finding better answers to
destabilize competing strategies}

To somewhat mitigate an issue related to the main difference between
best answers and best replies, we will play two rounds of the tournament
in our seminar. After the first round, the teams get the source code of
all other strategies and have some time to develop new strategies that
are just designed to become best answers against the strategies of the
other teams. More precisely, the goal of a team's second stage
strategies is just to increase the computed instability of other teams'
first stage strategies and thereby decrease their score. This will be
explained in more detail in the seminar.

\subsection{5. Guidelines for your Strategies}

\subsubsection{Keep it sufficiently simple, intuitive, clean and fast}

Your strategies and the implementation as R should be intuitively
understandable. It is not the goal to develop extremely complex
strategies with a large number of states that nobody understands and
that require extremely complicated computations. The idea is that
strategies resemble some sort of social norms. Also take into account
that for the simulations we may run your strategy on the order of a
million times, so don't make any complicated computations that need very
long to run. That being said your strategy does not have to be as simple
as tit-for-tat.

\subsubsection{Don't cheat}

If you are a hacker, you may think of many ways to cheat. For example,
it might be really useful to find out whether your strategy plays
against itself, but the rules of the basic PD game don't give you an
effective way to communicate (you can only choose C or D). So you may
think of exchanging information by writing information into a global R
variable which the other player can read out. Such things are considered
cheating and \textbf{not allowed}.

\begin{itemize}
\itemsep1pt\parskip0pt\parsep0pt
\item
  You are only allowed to use the information that is passed to the
  function as parameters (including the states you returned in earlier
  periods).
\item
  You are not allowed to modify any external objects.
\end{itemize}

As a rule of thumb: if you wonder whether something is cheating it
probably is; if you are not sure ask us.

\subsection{6. Debugging a strategy}

When you first write a strategy or other R function, it often does not
work correctly: your function is likely to have bugs. Some bugs make
your programm stop and throw an error message, other bugs are more
subtle and make your function run in a different fashion than you
expected. \emph{Debugging} means to find and correct bugs in your code.
There are different tools that help debugging. I want to illustrate some
debugging steps with an example.

Consider the following strategy, which I call ``exploiter'':

\begin{itemize}
\itemsep1pt\parskip0pt\parsep0pt
\item
  Exploiter: In the first period cooperate. If the other player
  cooperates for two or more times in a row defect. Otherwise play with
  70\% probability tit-for-tat and with 30\% probability play defect.
\end{itemize}

Here is a first attempt to implement this strategy as an r function (it
contains a lot of bugs):

\begin{verbatim}
exploiter = function(obs,i,t,game, otherC) {
  debug.store("exploiter",i,t) # Store each call for each player
  debug.restore("exploiter",i=1,t=2) # Restore call for player i in period t

  # Start nice in first period
  if (t=1) {
    return(list(a="C",otherC=0))
  }
  # If the other player has chosen C two or more times in a row play D
  if (obs$a[[j]]=="C") otherC= otherC + 1

  if (otherC > 2) return(list(a="D"))

  # Play tit for tat with probability 70% and with prob. 30% play D
  if (runif(1)<70) {
    a = obs$a[[j]]
  } else {
    a = "D"
  }
  return(nlist(a=a,otherC))
}
\end{verbatim}

\subsubsection{Step 1: Run function definition in R console and correct
errors}

As a first step select the whole function in the RStudio editor and
press the run button. You should see something similar to the following
in the R console.

\begin{verbatim}
> exploiter = function(obs,i,t,game, other.weakness) {
+  debug.store("exploiter",i,t) # Store each call for each player
+  debug.restore("exploiter",i=1,t=2) # Restore call for player i in period t
+  if (t=1) {
Error: unexpected '=' in:
"exploiter = function(obs,i,t,game, other.weakness) {
  if (t="
>     return(list(a="C",other.weakness=0))
Error: no function to return from, jumping to top level
>   }
Error: unexpected '}' in "  }"
>   if (obs$a[[j]]=="C") {
+     other.weakness = other.weakness + 1
+   }
Error: object 'j' not found
>   if (other.weakness > 2) {
+     return(list(a="D"))
+   }
Error: object 'other.weakness' not found
>   # Follow tit for tat with probability 70% otherwise play D
>   a = ifelse(runif(1)<0.7,obs$a[[j]],"D")
Error in ifelse(runif(1) < 0.7, obs$a[[j]], "D") : object 'j' not found
>   return(nlist(a=a,other.weakness))
Error in nlist(a = a, other.weakness) : object 'other.weakness' not found
> }
Error: unexpected '}' in "}"
\end{verbatim}

There are a lot of error messages. It is best to start with the first
error message and try to correct the corresponding code.

\begin{verbatim}
  if (t=1) {
  Error: unexpected '=' in:
  "exploiter = function(obs,i,t,game, other.weakness) {if (t="
\end{verbatim}

This is a typical beginner error. If we want to check whether t is 1, we
need to write t==1 instead of t=1. (The expression t=1 means that the
value 1 is assigned to the variable t, expression t==1 is a boolean
expression that is TRUE if t is 1 and FALSE otherwise.) A corrected
version of the function is

\begin{verbatim}
exploiter = function(obs, i, t, game, otherC) {
    debug.store("exploiter", i, t)  # Store each call for each player
    debug.restore("exploiter", i = 1, t = 2)  # Restore call for player i in period t

    # Start nice in first period
    if (t == 1) {
        return(list(a = "C", otherC = 0))
    }
    # If the other player has chosen C two or more times in a row play D
    if (obs$a[[j]] == "C") 
        otherC = otherC + 1

    if (otherC > 2) 
        return(list(a = "D"))

    # Play tit for tat with probability 70% and with prob. 30% play D
    if (runif(1) < 70) {
        a = obs$a[[j]]
    } else {
        a = "D"
    }
    return(nlist(a = a, otherC))
}
\end{verbatim}

If you run this new version in the console, no error is shown.
Unfortunately, this does not mean that there

\subsubsection{Step 2: Check whether run.rep.game yields errors and
debug such errors by stepping trough function}

As next step let us run run.rep.game with the strategy and check whether
some errors are shown.

\begin{verbatim}
run.rep.game(delta = 0.95, game = game, strat = nlist(exploiter, random.action))
\end{verbatim}

\begin{verbatim}
## Error in evaluating strategy exploiter in period t=2 for player i=1
## ERROR.HIST:
\end{verbatim}

\begin{verbatim}
##   obs_a1 obs_a2   a1   a2 pi1 pi2 otherC_1
## 1     NA     NA    C    D  -1   2        0
## 2     NA     NA <NA> <NA>  NA  NA       NA
\end{verbatim}

\begin{verbatim}
## Error: Error in (function (obs, i, t, game, otherC) : object 'j' not found
\end{verbatim}

We get an error message and learn that an error occurred when calling
exploiter for player i=1 in period t=2. We also get the error message
``object `j' not found''. Probably you see the problem directly from
that message. Nevertheless, let us pretend we have not found the problem
yet and let us step through our function. Go to the function code and
run the line

\begin{verbatim}
debug.restore("exploiter", i = 1, t = 2)  # Restore call for player i in period t
\end{verbatim}

\begin{verbatim}
## Restored:  game,i,obs,otherC,t
\end{verbatim}

in the R console by selecting the line and pressing the ``Run'' button
or Ctrl-Enter. This call now restores now the arguments with which the
strategy has been called for player i=1 in period t=2. You can examine
the function arguments by typing them in the R console:

\begin{verbatim}
obs
\end{verbatim}

\begin{verbatim}
## $a
##  a1  a2 
## "D" "D"
\end{verbatim}

\begin{verbatim}
i
\end{verbatim}

\begin{verbatim}
## [1] 1
\end{verbatim}

\begin{verbatim}
t
\end{verbatim}

\begin{verbatim}
## [1] 2
\end{verbatim}

You can also run some further lines of code inside the function to see
where exactly the error has occured:

\begin{verbatim}
# Start nice in first period
if (t == 1) {
    return(list(a = "C", otherC = 0))
}
# If the other player has chosen C two or more times in a row play D
if (obs$a[[j]] == "C") otherC = otherC + 1
\end{verbatim}

\begin{verbatim}
## Error: object 'j' not found
\end{verbatim}

We can also run parts of the last line to narrow down the error\ldots{}

\begin{verbatim}
obs$a[[j]]
\end{verbatim}

\begin{verbatim}
## Error: object 'j' not found
\end{verbatim}

\begin{verbatim}
j
\end{verbatim}

\begin{verbatim}
## Error: object 'j' not found
\end{verbatim}

Ok, clearly we forgot to define the variable j, which shall be the index
of the other player. We can add the line j = 3-i and run again the code
inside the corrected function:

\begin{verbatim}
debug.restore("exploiter", i = 1, t = 2)  # Restore call for player i in period t
\end{verbatim}

\begin{verbatim}
## Restored:  game,i,obs,otherC,t
\end{verbatim}

\begin{verbatim}
# Start nice in first period
if (t == 1) {
    return(list(a = "C", otherC = 0))
}
j = 3 - i  # index of other player

# If the other player has chosen C two or more times in a row play D
if (obs$a[[j]] == "C") otherC = otherC + 1
if (otherC > 2) return(list(a = "D"))

# Play tit for tat with probability 70% and with prob. 30% play D
if (runif(1) < 70) {
    a = obs$a[[j]]
} else {
    a = "D"
}
return(nlist(a = a, otherC))
\end{verbatim}

\begin{verbatim}
## $a
## [1] "D"
## 
## $otherC
## [1] 0
\end{verbatim}

You probably will see an error message after the last line that there is
no function to return from, but we can ignore that one. Otherwise we see
no more error. Yet, that does not mean that our function has no more
bug. Before proceeding we copy the whole corrected function definition
into the R console:

\begin{verbatim}
exploiter = function(obs, i, t, game, otherC) {
    debug.store("exploiter", i, t)  # Store each call for each player
    debug.restore("exploiter", i = 1, t = 2)  # Restore call for player i in period t

    # Start nice in first period
    if (t == 1) {
        return(list(a = "C", otherC = 0))
    }
    j = 3 - i  # index of other player

    # If the other player has chosen C two or more times in a row play D
    if (obs$a[[j]] == "C") 
        otherC = otherC + 1
    if (otherC > 2) 
        return(list(a = "D"))

    # Play tit for tat with probability 70% and with prob. 30% play D
    if (runif(1) < 70) {
        a = obs$a[[j]]
    } else {
        a = "D"
    }
    return(nlist(a = a, otherC))
}
\end{verbatim}

\subsubsection{Step 3: Running run.rep.game again and debugging the next
error}

Copy the corrected function in your R console and then call run.rep.game
again. (Note I now call the function run.rep.game with the parameters
game.seed and strat.seed, which ensure that the random generator always
returns the same results. That is just for the reason that it is easier
to write this documentation if the error always occures in the same
period).

\begin{verbatim}
run.rep.game(delta = 0.95, game = game, strat = nlist(exploiter, random.action), 
    game.seed = 12345, strat.seed = 12345)
\end{verbatim}

\begin{verbatim}
## Error in evaluating strategy exploiter in period t=8 for player i=1
## ERROR.HIST:
\end{verbatim}

\begin{verbatim}
##   obs_a1 obs_a2   a1   a2 pi1 pi2 otherC_1
## 3      D      D    D    C   2  -1        0
## 4      D      C    C    C   1   1        1
## 5      D      C    C    D  -1   2        2
## 6      C      D    D    C   2  -1        2
## 7      D      C    D    C   2  -1       NA
## 8   <NA>   <NA> <NA> <NA>  NA  NA       NA
\end{verbatim}

\begin{verbatim}
## Error: Error in otherC + 1: 'otherC' is missing
\end{verbatim}

We find an error in period t=8 . Let us investigate the call to our
strategy in that period by setting t=8 in the call to debug.restore

\begin{verbatim}
debug.restore("exploiter", i = 1, t = 8)  # Restore call for player i in period t
\end{verbatim}

\begin{verbatim}
## Variable  otherC  was missing.
## Restored:  game,i,obs,otherC,t
\end{verbatim}

The call tells me that the state variable otherC was not provided as an
argument to this function. This basically means that in period t=7 the
function did not return the variable otherC. Let us check where this
problem happened by exploring in more detail the function call in period
7.

\begin{verbatim}
debug.restore("exploiter", i = 1, t = 7)  # Restore call for player i in period t
\end{verbatim}

\begin{verbatim}
## Restored:  game,i,obs,otherC,t
\end{verbatim}

\begin{verbatim}
# Start nice in first period
if (t == 1) {
    return(list(a = "C", otherC = 0))
}
j = 3 - i  # index of other player

# If the other player has chosen C two or more times in a row play D
if (obs$a[[j]] == "C") otherC = otherC + 1
if (otherC > 2) return(list(a = "D"))
\end{verbatim}

\begin{verbatim}
## $a
## [1] "D"
\end{verbatim}

\begin{verbatim}
Error: no function to return from, jumping to top level
\end{verbatim}

We see that the function returned in the last line of the code above.
And of course, we forgot to add otherC in the list of returned
variables. So this variable was missing in period t=8. The last line is
easy to fix and we again paste into the R console a corrected version of
our strategy:

\begin{verbatim}
exploiter = function(obs, i, t, game, otherC) {
    debug.store("exploiter", i, t)  # Store each call for each player
    debug.restore("exploiter", i = 1, t = 2)  # Restore call for player i in period t

    # Start nice in first period
    if (t == 1) {
        return(list(a = "C", otherC = 0))
    }
    j = 3 - i  # index of other player

    # If the other player has chosen C two or more times in a row play D
    if (obs$a[[j]] == "C") 
        otherC = otherC + 1
    if (otherC > 2) 
        return(list(a = "D", otherC = otherC))

    # Play tit for tat with probability 70% and with prob. 30% play D
    if (runif(1) < 70) {
        a = obs$a[[j]]
    } else {
        a = "D"
    }
    return(nlist(a = a, otherC))
}
\end{verbatim}

\subsubsection{Step 4: Call run.rep.game again and remove remaining
bugs}

\begin{verbatim}
run.rep.game(delta = 0.95, game = game, strat = nlist(exploiter, random.action), 
    game.seed = 12345, strat.seed = 12345)
\end{verbatim}

\begin{verbatim}
## $hist
##    obs_a1 obs_a2 a1 a2 pi1 pi2 otherC_1
## 1    <NA>   <NA>  C  D  -1   2        0
## 2       C      D  D  D   0   0        0
## 3       D      D  D  C   2  -1        0
## 4       D      C  C  C   1   1        1
## 5       D      C  C  D  -1   2        2
## 6       C      D  D  C   2  -1        2
## 7       D      C  D  C   2  -1        3
## 8       D      C  D  D   0   0        4
## 9       D      D  D  C   2  -1        4
## 10      D      C  D  C   2  -1        5
## 11      D      C  D  C   2  -1        6
## 12      D      C  D  C   2  -1        7
## 13      D      C  D  C   2  -1        8
## 14      D      C  D  C   2  -1        9
## 15      D      D  D  D   0   0        9
## 16      D      D  D  C   2  -1        9
## 17      D      C  D  C   2  -1       10
## 18      D      C  D  D   0   0       11
## 19      D      D  D  D   0   0       11
## 20      D      D  D  D   0   0       11
## 21      D      D  D  C   2  -1       11
## 22      D      C  D  D   0   0       12
## 23      D      D  D  D   0   0       12
## 
## $u
## [1]  1.0000 -0.3043
\end{verbatim}

There is no more error message but there are still 2 bugs left in the
function such that the programmed strategy does not work as verbally
described. (Remember that the strategy shall only automatically defect
if at last two times \textbf{in a row} the other player has played C). I
will leave the debugging of the last function as an exercise. I just end
with a little debugging hint. It can be useful to define additional
helper states so that one gets better information in the result table.
For example, I add a state ``played.tit.for.tat'' that is TRUE if the
function exploiter indeed played tit.for.tat in the current round and
otherwise will be shown as NA: (For the state to appear in the table, it
must be returned in the first period)

\begin{verbatim}
exploiter = function(obs, i, t, game, otherC, played.tit.for.tat) {
    debug.store("exploiter", i, t)  # Store each call for each player
    debug.restore("exploiter", i = 1, t = 2)  # Restore call for player i in period t

    # Start nice in first period
    if (t == 1) {
        return(list(a = "C", otherC = 0))
    }
    j = 3 - i  # index of other player

    # If the other player has chosen C two or more times in a row play D
    if (obs$a[[j]] == "C") 
        otherC = otherC + 1
    if (otherC > 2) 
        return(list(a = "D", otherC = otherC))

    # Play tit for tat with probability 70% and with prob. 30% play D
    if (runif(1) < 70) {
        a = obs$a[[j]]
        played.tit.for.tat = TRUE
    } else {
        a = "D"
        played.tit.for.tat = NA
    }
    return(nlist(a = a, otherC, played.tit.for.tat))
}

run.rep.game(delta = 0.95, game = game, strat = nlist(exploiter, random.action), 
    game.seed = 12345, strat.seed = 12345)
\end{verbatim}

\begin{verbatim}
## $hist
##    obs_a1 obs_a2 a1 a2 pi1 pi2 otherC_1
## 1    <NA>   <NA>  C  D  -1   2        0
## 2       C      D  D  D   0   0        0
## 3       D      D  D  C   2  -1        0
## 4       D      C  C  C   1   1        1
## 5       D      C  C  D  -1   2        2
## 6       C      D  D  C   2  -1        2
## 7       D      C  D  C   2  -1        3
## 8       D      C  D  D   0   0        4
## 9       D      D  D  C   2  -1        4
## 10      D      C  D  C   2  -1        5
## 11      D      C  D  C   2  -1        6
## 12      D      C  D  C   2  -1        7
## 13      D      C  D  C   2  -1        8
## 14      D      C  D  C   2  -1        9
## 15      D      D  D  D   0   0        9
## 16      D      D  D  C   2  -1        9
## 17      D      C  D  C   2  -1       10
## 18      D      C  D  D   0   0       11
## 19      D      D  D  D   0   0       11
## 20      D      D  D  D   0   0       11
## 21      D      D  D  C   2  -1       11
## 22      D      C  D  D   0   0       12
## 23      D      D  D  D   0   0       12
## 
## $u
## [1]  1.0000 -0.3043
\end{verbatim}

\subsubsection{Exercise: Correct the remaining bugs in exploiter}

That was the tutorial. Take at the look at the upcoming problem sets
that will describe the tournament tasks\ldots{}

\end{document}
